% Options for packages loaded elsewhere
\PassOptionsToPackage{unicode}{hyperref}
\PassOptionsToPackage{hyphens}{url}
      %
\documentclass[
      11pt,
                      ]{article}
        \usepackage{lmodern}
    \usepackage{amssymb,amsmath}
\usepackage{ifxetex,ifluatex}
\ifnum 0\ifxetex 1\fi\ifluatex 1\fi=0 % if pdftex
\usepackage[T1]{fontenc}
\usepackage[utf8]{inputenc}
\usepackage{textcomp} % provide euro and other symbols
\else % if luatex or xetex
  \usepackage{unicode-math}
  \defaultfontfeatures{Scale=MatchLowercase}
\defaultfontfeatures[\rmfamily]{Ligatures=TeX,Scale=1}
  \setmainfont[]{Times New Roman}
    \setsansfont[]{Quicksand}
              \fi
  % Use upquote if available, for straight quotes in verbatim environments
\IfFileExists{upquote.sty}{\usepackage{upquote}}{}
\IfFileExists{microtype.sty}{% use microtype if available
  \usepackage[]{microtype}
  \UseMicrotypeSet[protrusion]{basicmath} % disable protrusion for tt fonts
}{}
  \makeatletter
\@ifundefined{KOMAClassName}{% if non-KOMA class
  \IfFileExists{parskip.sty}{%
    \usepackage{parskip}
  }{% else
    \setlength{\parindent}{0pt}
    \setlength{\parskip}{6pt plus 2pt minus 1pt}
  }
}{% if KOMA class
  \KOMAoptions{parskip=half}}
\makeatother
    \usepackage{xcolor}
\IfFileExists{xurl.sty}{\usepackage{xurl}}{} % add URL line breaks if available
%\urlstyle{same} % disable monospaced font for URLs
\usepackage{url}
      \usepackage[margin=1in]{geometry}
                                          \setlength{\emergencystretch}{3em} % prevent overfull lines
      \providecommand{\tightlist}{%
        \setlength{\itemsep}{0pt}\setlength{\parskip}{0pt}}
              \setcounter{secnumdepth}{-\maxdimen} % remove section numbering
                                                          
                        \ifluatex
          \usepackage{selnolig}  % disable illegal ligatures
          \fi
                                                                          \newlength{\cslhangindent}
              \setlength{\cslhangindent}{1.5em}
              \newenvironment{cslreferences}%
              {}%
              {\par}
                              
                                \title{RESEARCH PROPOSAL PROFILES OF
TOLERANCE AND RESPECT FOR THE RIGHTS OF DIVERSE}
                                                  \author{true}
                  \date{November 30, 2020}
                                      
                    % Jesus, okay, everything above this comment is default Pandoc LaTeX template. -----
                    % ----------------------------------------------------------------------------------
                    % I think I had assumed beamer and LaTex were somehow different templates.
                  
                  
                  \usepackage{kantlipsum}
                  
                  \usepackage{abstract}
                  \renewcommand{\abstractname}{}    % clear the title
                  \renewcommand{\absnamepos}{empty} % originally center
                  
                  \renewenvironment{abstract}
                  {{%
                    \setlength{\leftmargin}{0mm}
                    \setlength{\rightmargin}{\leftmargin}%
                  }%
                    \relax}
                  {\endlist}
                  
                  \makeatletter
                  \def\@maketitle{%
                    \newpage
                    %  \null
                    %  \vskip 2em%
                      %  \begin{center}%
                      \let \footnote \thanks
                                          {\fontsize{14.5}{20}\selectfont\bfseries\sffamily\raggedright  \setlength{\parindent}{0pt} \@title \par}%
                                      }
                  %\fi
                  \makeatother
                  
                  
                                      \title{RESEARCH PROPOSAL PROFILES
OF TOLERANCE AND RESPECT FOR THE RIGHTS OF DIVERSE }
                   
                    
                    
                    
                    %\author{\Large Pamela Inostroza
Fernandez\vspace{0.05in} \newline\normalsize\emph{KU LEUVEN}  }
                    
                    
                    \date{}
                  
                  \usepackage{titlesec}
                  
                  % \sffamily\uppercase
                    \titleformat*{\section}{\large\bfseries\sffamily\uppercase}
                  \titleformat*{\subsection}{\bfseries\sffamily} % \small\uppercase
                  \titleformat*{\subsubsection}{\normalsize\itshape}
                  \titleformat*{\paragraph}{\normalsize\itshape}
                  \titleformat*{\subparagraph}{\normalsize\itshape}
                  
                  % add some other packages ----------
                    
                    % \usepackage{multicol}
                  % This should regulate where figures float
                  % See: https://tex.stackexchange.com/questions/2275/keeping-tables-figures-close-to-where-they-are-mentioned
                  \usepackage[section]{placeins}
                  
                  
                  
                  \makeatletter
                  \@ifpackageloaded{hyperref}{}{%
                    \ifxetex
                    \PassOptionsToPackage{hyphens}{url}\usepackage[setpagesize=false, % page size defined by xetex
                                                                   unicode=false, % unicode breaks when used with xetex
                                                                   xetex]{hyperref}
                    \else
                      \PassOptionsToPackage{hyphens}{url}\usepackage[draft,unicode=true]{hyperref}
                    \fi
                  }
                  
                  \@ifpackageloaded{color}{
                    \PassOptionsToPackage{usenames,dvipsnames}{color}
                  }{%
                    \usepackage[usenames,dvipsnames]{color}
                  }
                  \makeatother
                  \hypersetup{breaklinks=true,
                  bookmarks=true,
                  pdfauthor={Pamela Inostroza Fernandez (KU LEUVEN)},
                  pdfkeywords = {latent class analysis, large scale
assessment, r, categorical analysis, latent profile analysis},  
                  pdftitle={RESEARCH PROPOSAL PROFILES OF TOLERANCE AND
RESPECT FOR THE RIGHTS OF DIVERSE},
                  colorlinks=true,
                  citecolor=blue,
                  urlcolor=blue,
                  linkcolor=magenta,
                  pdfborder={0 0 0}}
                  \usepackage{url}
                  %\urlstyle{same}  % don't use monospace font for urls
	       \usepackage{hanging}
% Add an option for endnotes. -----



% This will better treat References as a section when using natbib
% https://tex.stackexchange.com/questions/49962/bibliography-title-fontsize-problem-with-bibtex-and-the-natbib-package



% set default figure placement to htbp
\makeatletter
\def\fps@figure{htbp}
\makeatother




\newtheorem{hypothesis}{Hypothesis}
\begin{document}

%\textsf{\textbf{This is sans-serif bold text.}}
%\textbf{\textsf{This is bold sans-serif text.}}


% \maketitle

{% \usefont{T1}{pnc}{m}{n}
\setlength{\parindent}{0pt}
\thispagestyle{plain}
{%\fontsize{18}{20}\selectfont\raggedright
\maketitle  % title \par

}




{
   \vskip 13.5pt\relax \normalsize\fontsize{11}{12} 
   \MakeUppercase{\textsf{\large Pamela Inostroza Fernandez}}, \small{KU
LEUVEN}   

}

}








\begin{abstract}

%    \hbox{\vrule height .2pt width 39.14pc}

    \vskip 8.5pt % \small 

\noindent \small{kdnkdnfke}


\vskip 8.5pt \noindent \emph{Keywords}: latent class analysis, large
scale assessment, r, categorical analysis, latent profile analysis \par

%    \hbox{\vrule height .2pt width 39.14pc}



\end{abstract}


\vskip -8.5pt

{
\hypersetup{linkcolor=black}
\setcounter{tocdepth}{2}
\tableofcontents
}

 % removetitleabstract

{
\setcounter{tocdepth}{2}
\tableofcontents
}
\setlength{\parindent}{16pt}
\setlength{\parskip}{0pt}

\hypertarget{introduction}{%
\section{INTRODUCTION}\label{introduction}}

The development of civic values and attitudes of tolerance and respect
for the rights of diverse social groups among youth are essential for
sustainable democratic societies. These values are strongly promoted by
families, educational systems and international organizations across the
world. The measurements and comparison of these attitudes among youth
can provide valuable information about their development in different
societies and over time.\\
\newline   Same international studies such as the International Civic
and Citizenship Education Study (ICCS) provide extensive comparative
information regarding these aspects. The ICCS study is a large-scale
assessment (survey) applied in more than 25 educational systems during
the last three cycles and focused on secondary education (representative
samples of 8th graders, 14-year-olds in each country) addressing topics
such as citizenship, diversity and social interactions at school. The
study produces internationally comparative data collected via student,
school and teacher questionnaires. Data from different waves of the ICCS
survey is publicly available to researchers. The first time this study
was applied was in 1999 to 28 countries and it was called CIVED, the
second wave started using the name ICCS and was implemented in 2009 in
38 countries, the last study was performed in 2016 to 24 countries. The
next cycle is scheduled for 2022 and 25 countries will participate.\\
\newline  Previous research using ICCS data has been largely focused on
average country comparisons of attitudinal measures such as attitudes
toward equal rights for immigrants, ethnic minorities and women, norms
of good citizenship behaviour and political participation. Most of these
studies employed variable-centered analyses. Nevertheless, recent
studies started to show the usefulness of person-centered approaches
(i.e.~latent class analysis, hereafter LCA) aimed at identifying
profiles of young people's attitudes. For example, using ICCS 2009 data,
(Hooghe, Oser, and Marien 2016) compare profiles of good citizenship
norms across 38 countries and distinguished distinctive subgroups of the
population that share a common understanding of what constitutes good
citizenship were identified (e.g.~who express either engaged or
duty-based citizenship norms).\\
\newline  Another study focused their research on changes over time
(where the research design and data gathering methods are strictly
comparable) (Hooghe and Oser 2015). For this, CIVED 1999 and ICCS 2009
was used. The scope of the analysis was threefold. First, distinct
profiles of good citizenship norms were identified in both cycles.
Second, trends over time were investigated and finally, differences
between countries and/over time were analysed in detail. Nevertheless,
most of these studies employing LCA with ICCS data focused on patterns
within a particular type of attitude described by individual items
(e.g.~citizenship norms) leaving space for investigations that aim to
capture a wider set of attitudinal measures described by scores on
different variables.\\
\newline  To address this gap, this research will approach the topic of
tolerance and respect for the rights of diverse social groups
operationalized as a multifaceted set of attitudes toward equal rights
for immigrants, ethnic minorities and women. This topic was addressed by
previous studies aimed at comparing these attitudinal measures mostly in
isolation across countries and over time. However, to date, no studies
addressed the potential interdependence of these three attitudinal
dimensions among different subgroups of people (e.g.~highly tolerant,
highly intolerant regarding all aspects, etc.). Therefore, the current
study aims to fill this gap by addressing the following research
questions:\\
\newline
1. What profiles of tolerance and respect for the rights of diverse
social groups are observed among adolescents in different countries?\\
2. Are these profiles comparable across countries and over time?\\
3. What individual and contextual factors are associated with profile
membership? Do they vary depending on the context of the country or the
cohort?

\hypertarget{data}{%
\section{DATA}\label{data}}

Multiple countries had participated in the ICCS study during the last
three cycles (detailed participation of the selected countries can be
found in Table 1 in the Annex). Some of the participating countries can
be classified by the following grouping\footnote{The exact amount of
  countries and cycles will be defined during the thesis project
  development. The same will occur for the amount of scales and
  variables that will be included}:\\
\newline  a) Nordic Countries: Denmark, Finland, Norway, Sweden.\\
b) Western European Countries: Belgium (Flemish), The Netherlands.\\
c) Central and Eastern European Countries: Bulgaria, Estonia, Latvia,
Lithuania, Croatia, Slovenia.\\
d) Southern European Countries: Italy, Malta.\\
e) Latin American Countries: Chile, Peru, Colombia, Dominican Republic,
Mexico\\
\newline  Each student participating in the study was received a test
tapping into his civic knowledge and skills and obtained a
score\footnote{Scores were calculated through multiple imputation for
  ICCS 2009 and ICCS 2016, this means five plausible values are
  available.}. Moreover, background questionnaires were administered to
capture students' perceptions and attitudes toward civic and
citizenship, including attitudes toward equal rights for immigrants,
ethnic minorities and women. Databases include not only the responses to
individual items but also indexes for the scales that were constructed.
This research will be focused in three indexes called in the last cycle
as ``Attitudes toward equal rights for immigrants,'' ``Attitudes toward
gender equality'' and ``Attitudes toward equal rights for all
ethnic/racial groups.'' Each item and the respective construct evaluated
is detailed in Table 2, Table 3 and Table 4 for each cycle.

\hypertarget{methods}{%
\section{METHODS}\label{methods}}

All cycles of ICCS (CIVED) have been validated through variable-centred
analysis, this means that latent constructs and the invariance across
countries have been consistently validated thoroughly using CFA. On the
contrary, not many research has been done using person-centred
approaches, as Latent Profile Analysis (LPA) and Latent Class analysis
(LCA).\\
\newline  The latent class model assumes the existence of a latent
categorical variable such that the observed response variables are
conditionally independent, given that variable. LCA treat a contingency
table as a finite mixture of unobserved tables generated under a
conditional independence structure of a latent variable (Agresti 2013).
In other words, LCA can directly assess the theory that distinctive
groups of people share specific attitudes. Depending on the response
variable in the model the analysis is called Latent Profile Analysis if
is Continuous (Normal) and Latent Class Analysis if the response
variable is Categorical (Multinomial).\\
\newline  In LCA, studying measurement invariance is necessary to
determine whether the number and nature of the latent profiles are the
same across the different observed groups (Olivera-Aguilar and Rikoon
2018). For this, multiple group LCA models are computed, and the
relative fit of the unconstrained and semi-constrained models are
compared using the LRT, AIC, BIC, and aBIC measures. Also is needed to
review any kind of response bias, the most common refers to ``a
systematic tendency to respond to a range of questionnaire items on some
basis other than the specific item content'' for example e.g.~extreme or
agree/disagree (Kankaraš, Vermunt, and Moors 2011).\\
\newline  In order to assess the cross-national and cross-cohort
comparability using CFA, new scales should be created that fit across
all countries and cohorts analysed, rather than using the ones already
created by the consortium\footnote{In CIVED 1999 index MINORMLE was not
  include in the datasets.} (Barber and Ross 2020).\\
\newline  Descriptive and main analysis can be performed in R software
using lavaan and poLCA/lcca packages (Robertson and Kaptein 2016). Most
complex analysis could be implemented in LEM. If necessary, it may be
possible to access to a licence for MPLUS or Latent Gold software.

\newpage

\hypertarget{references}{%
\section*{REFERENCES}\label{references}}
\addcontentsline{toc}{section}{REFERENCES}

\hypertarget{refs}{}
\begin{CSLReferences}{1}{0}
\leavevmode\hypertarget{ref-agresti_categorical_2013}{}%
Agresti, Alan. 2013. \emph{Categorical Data Analysis}. 3rd ed. Wiley
Series in Probability and Statistics 792. Hoboken, NJ: Wiley.

\leavevmode\hypertarget{ref-barber_profiles_2020}{}%
Barber, Carolyn, and Jessica Ross. 2020. {``Profiles of Adolescents'
Civic Attitudes in Sixteen Countries: {Examining} Cross-Cohort Changes
from 1999 to 2009.''} \emph{Research in Comparative and International
Education} 15 (2): 79--96.
\url{https://doi.org/10.1177/1745499920910583}.

\leavevmode\hypertarget{ref-hooghe_rise_2015}{}%
Hooghe, Marc, and Jennifer Oser. 2015. {``The Rise of Engaged
Citizenship: {The} Evolution of Citizenship Norms Among Adolescents in
21 Countries Between 1999 and 2009.''} \emph{International Journal of
Comparative Sociology} 56 (1): 29--52.
\url{https://doi.org/10.1177/0020715215578488}.

\leavevmode\hypertarget{ref-hooghe_comparative_2016}{}%
Hooghe, Marc, Jennifer Oser, and Sofie Marien. 2016. {``A Comparative
Analysis of {`Good Citizenship'}: {A} Latent Class Analysis of
Adolescents' Citizenship Norms in 38 Countries.''} \emph{International
Political Science Review} 37 (1): 115--29.
\url{https://doi.org/10.1177/0192512114541562}.

\leavevmode\hypertarget{ref-isac_indicators_2019}{}%
Isac, Maria Magdalena, Laura Palmerio, and M. P. C. (Greetje) van der
Werf. 2019. {``Indicators of (in)tolerance Toward Immigrants Among
{European} Youth: An Assessment of Measurement Invariance in {ICCS}
2016.''} \emph{Large-Scale Assessments in Education} 7 (1): 6.
\url{https://doi.org/10.1186/s40536-019-0074-5}.

\leavevmode\hypertarget{ref-kankaras_measurement_2011}{}%
Kankaraš, Miloš, Jeroen K. Vermunt, and Guy Moors. 2011. {``Measurement
{Equivalence} of {Ordinal} {Items}: {A} {Comparison} of {Factor}
{Analytic}, {Item} {Response} {Theory}, and {Latent} {Class}
{Approaches}.''} \emph{Sociological Methods \& Research} 40 (2):
279--310. \url{https://doi.org/10.1177/0049124111405301}.

\leavevmode\hypertarget{ref-sandoval-hernandez_measurement_2018}{}%
Miranda, Daniel, and Juan Carlos Castillo. 2018. {``Measurement {Model}
and {Invariance} {Testing} of {Scales} {Measuring} {Egalitarian}
{Values} in {ICCS} 2009.''} In \emph{Teaching {Tolerance} in a
{Globalized} {World}}, edited by Andrés Sandoval-Hernández, Maria
Magdalena Isac, and Daniel Miranda, 4:19--31. Cham: Springer
International Publishing.
\url{https://doi.org/10.1007/978-3-319-78692-6_3}.

\leavevmode\hypertarget{ref-munck_measurement_2018}{}%
Munck, Ingrid, Carolyn Barber, and Judith Torney-Purta. 2018.
{``Measurement {Invariance} in {Comparing} {Attitudes} {Toward}
{Immigrants} {Among} {Youth} {Across} {Europe} in 1999 and 2009: {The}
{Alignment} {Method} {Applied} to {IEA} {CIVED} and {ICCS}.''}
\emph{Sociological Methods \& Research} 47 (4): 687--728.
\url{https://doi.org/10.1177/0049124117729691}.

\leavevmode\hypertarget{ref-olivera-aguilar_assessing_2018}{}%
Olivera-Aguilar, Margarita, and Samuel H. Rikoon. 2018. {``Assessing
{Measurement} {Invariance} in {Multiple}-{Group} {Latent} {Profile}
{Analysis}.''} \emph{Structural Equation Modeling: A Multidisciplinary
Journal} 25 (3): 439--52.
\url{https://doi.org/10.1080/10705511.2017.1408015}.

\end{CSLReferences}

\end{document}
